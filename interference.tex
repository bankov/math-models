{
\newcommand{\prej}[1]{p_{rej}^{(#1)}}
\newcommand{\Est}[1]{\mathbb{E}_{#1}[SendTime]}
\newcommand{\tst}[2]{\tau_{#1}^{(#2)}}
\newcommand{\pc}[2]{p_{c_{#1}}^{(#2)}}
\newcommand{\at}[2]{a_{#1}^{(#2)}}
\newcommand{\lc}[1]{l_{c}^{(#1)}}
\newcommand{\ls}[1]{l_{s}^{(#1)}}
\newcommand{\la}[1]{l_{a}^{(#1)}}
\newcommand{\tslot}[1]{t_{slot}^{(#1)}}

\section{Аналитическая модель интерференции}
\begin{itemize}
\item пропускная способность $S_m$,
\item вероятность $\prej{m}$ сброса пакета ввиду достижения максимального числоа попыток,
\item среднее время $\Est{m}$ передачи одного пакета.
\end{itemize}
Эти показатели взаимосвязаны посредством формулы:
\begin{equation}
S_m = \frac{L(1-\prej{m})}{\Est{m}}.
\end{equation}

\section{Общий метод}
Пусть $\tst{m}{n}$ --- вероятность того, что станция $m$ начинает свою передачу в начале виртуального слота станции $n$. Пусть $\pc{k-1}{m}$ --- вероятность того, что произошла $k$-ая коллизия при передаче пакета станцией $m$, при условии, что уже произошли все предыдущие $k-1$ коллизий. В условиях отсутсвтия коллизий кадров данных среднее время передачи одного пакета определеяется следующим образом:
\begin{equation}
\Est{m} = \at{0}{m} + \sum\limits_{i = 1}^{N_r - 1} \at{i}{m}
				\prod\limits_{i=0}^{i-1}\pc{k}{m},
\end{equation} 
где $N_r$ --- максимальное число коротких передач, $\at{i}{m}$ --- длительность $i+1$-ой попытки, включая время передачи, а также время отсрочки, состоящее в среднем из $\frac{W_i-1}{2}$ виртуальных слотов:
\begin{equation}
\at{i}{m} = \frac{W_i-1}{2}\tslot{m} + (1-\pc{i}{m})l_s + \pc{i}{m} \lc{m}
\end{equation}
Здесь  $l_s = RTS + 3\times SIFS + CTS + DATA + ACK + DIFS$ --- время успешной передачи, $\lc{m}$ --- среднее время коллизии, а $\tslot{m}$ --- средняя длительность виртуального слота:
\begin{equation}
\tslot{m} = (1 - \tst{n}{m})\sigma  + \tst{n}{m}\la{m},
\end{equation}
где $\la{m}$ --- средняя длительность интервала, на который станция $m$ замораживает свой счетчик отсрочки из-за передачи другой станции, т.е. средняя длительность <<занятого>> слота.

При условии, что обмен кадрами RTS и CTS надежно защищает передачу данных, вероятность сброса пакета равна вероятности, что произойдет ровно $N_r$ коллизий:
\begin{equation}
\prej{m} = \pc{0}{m} \pc{1}{m}\dots\pc{N_r-1}{m}.
\end{equation}
\begin{equation}
f_m = 1 + \sum\limits_{i=1}^{N_r-1}\prod\limits_{j=0}^{i-1}\pc{j}{m} \text{ и } w_m = \frac{W_0-1}{2} + \sum\limits_{i=1}^{N_r-1}\frac{W_i-1}{2} \prod\limits_{j=0}^{i-1}\pc{j}{m}.
\end{equation}

\subsection{Случай 1}
Начнем с простейшего случая, когда все станции находятся в пределах TX-области (Transmission range) друг друга (Рис 2.), т.~е. любой кадр, переданный одной из станций, успешно принимается любой другой станцией в отсутствие коллизий. В данном случае $\la{m} = l_s$ и $\tst{1}{3} = \tst{3}{1} = \tau$ из соображений симметрии.

После успешной передачи станции 3 (см. рис. 3) станции 1 и 3 ждут DIFS и продолжают отсчитывать свои таймеры отсрочки. Если станции 1 и 3 начинают передавать одновременно, происходит коллизия и обе станции вынуждены ждать CTS-таймаут $T_c$ перед тем, как вернуться к процедуре отсчета интервала отсрочки. Поэтому, $\lc{m} = RTS + T_c$ и 
\begin{equation}
\label{case1:1}
\pc{i}{m} = \tau,\ i = \overline{0,N_r-1}.
\end{equation}
Подставляя \eqref{case1:1} в (4), найдем $\tau$ из (5):
\begin{equation}
\notag
\tau = \frac{f_m(\tau)}{w_m(\tau)+f_m(\tau)}.
\end{equation}
Далее, подставляя найденное значение $\tau$ в оставшиеся уравнения, получаем искомые величины.

\subsection{Случай 2}
Теперь рассмотрим случай 2, при котором станции 1 и 4 находятся в CS-области (Carrier Sense range) друг друга (см. рис. 4), т.е. любой кадр, переданный одной станцией, принимается другой станцией некорректно. В силу асимметричности данного случая $\tst{1}{3}$ и $\tst{3}{1}$  будут различны.

Сначала оценим пропускную способность соединения $1\rightarrow 2$. На рис. 5 показано, что после того, как станция 3 получила кадр ACK подтверждения от станции 4, она должна подождать время DIFS, в то время как станция 1 не может декодировать этот кадр и должна ждать время EIFS (EIFS > DIFS). Таким образом, возникает так называемый защитный интервал (ЗИ), внутри которого станция 3 может начать свою передачу, не соревнуясь за канал со станцией 1. Длина ЗИ в слотах равна $d = \lfloor \hat{d} \rfloor$, где $\hat{d} = \frac{EIFS-DIFS}{\sigma}$, а $d$ --- целая часть $\hat{d}$. Вероятность того, что станция 3 начнет свою передачу внутри ЗИ, равна $p_d = \frac{d+1}{W_0}$. 

Пусть станция 3 начала передавать внутри ЗИ. После этой передачи ситуация с ЗИ повторяется, причем такая цепочка последовательных передач может быть потенциально бесконечной и в среднем содержит $\frac{p_d}{1-p_d}$ дополнительных передач станции 3, выполняемых без конкуренции за канал со станцией 1, что является причиной неравномерного распределения пропускной способности между соединениями. Таким образом, выражение для $\la{1}$ будет следующим:
\begin{equation}
\la{1} = l_s + \hat{d}\sigma + \frac{p_d}{1-p_d}(l_s+\frac{d}{2}\sigma),
\end{equation}
где $l_s$ соответствует первому переданному станцией 3 пакету, при условии, что станция 1 и 3 были синхронизированы; $\hat{d}\sigma$ --- время после последней передачи пакета станцией 3, в течение которого станция 1 будет не будет передавать (это время равно EIFS - DIFS).

Значение $\hat{d}$ не является целым, поэтому после передачи станции 3 станции 1 и 3 отсчитывают свои интервалы отсрочки асинхронно и, следовательно, не могут вступить в коллизию (время RX/TX переключения станции из режима приема в режим передачи пренебрежимо мало). Они вновь синхронизируются  после успешной передачи станции 1 или коллизии. Поэтому попытка передачи станции 1 будет неудачной, если станция 3 не передает в течение интервала отсрочки станции 1 и начинает передачу одновременно со станцией 1. Вероятность того, что станция 1 выберет слот под номером $b$ и этот слот окажется коллизионным равна
\begin{equation}
\tst{3}{1}\frac{(1-\tst{3}{1})^b}{W_i}.
\end{equation}
Просуммировав по всем $b$ в результате имеем следующую формулу для вероятности коллизии:
\begin{equation}
\pc{i}{1} = \tst{3}{1} \frac{1}{W_i}\sum\limits_{b = 0}^{W_i-1}(1-\tst{3}{1})^b = 
\frac{1 - (1-\tst{3}{1})^{W_i}}{W_i}, i = \overline{0,N-1}.
\end{equation}
Зная $\tst{3}{1}$, находим показатели производительности соединения $1 \rightarrow 2$, используя (1), (2) и (3).

\textbf{Вопрос} Если станция 1 передает впервые, то возможные 2 ситуации: либо станции 1 и 3 синхронны, либо --- асинхронны. Станции синхронны тогда и только тогда, когда последней передавала станция 1, причем это передача могла быть как успешной, так и коллизионной. В случае, если станции синхронизированы, то вероятность коллизии при первой передаче равна $\pc{0}{1} = \dots$ . Если же станции не синхронизированы, то $\pc{0}{1} = 0$.

Теперь перейдем к оценке производительности соединения $3 \rightarrow 4$. Во-первых, необходимо учесть ЗИ. При первой попытке передачи пакета станция 3 не испытывает
конкуренции со стороны станции 1 с вероятностью $p_d$. Поэтому, с вероятностью  $p_d$ станция 3 отсчитает в среднем $\frac{d}{2}$ пустых слотов и успешно передаст свой пакет. С вероятностью $1-p_d$ станция 3 начнет передачу вне ЗИ. Она отсчитает $d$ пустых слотов и в среднем $\frac{W_0 - d - 1}{2}$ виртуальных слотов, которые могут быть заполнены передачами станции 1. Следовательно, формула для длительности первой попытки выглядит следующим образом:
\begin{equation}
\at{0}{3} = p_d(l_s + \frac{d}{2}\sigma) + (1 - p_d) \left[d\sigma + \frac{W_0-d-1}{2}\tslot{3}\right] + (1-\pc{0}{3} - p_d)l_s + \pc{0}{3}l_c,
\end{equation}
где $l_c = RTS + T_c$. 

Как уже отмечалось, после передачи станции 3 станции 1 и 3 асинхронны. Если в один из виртуальных слотов станции 3 станция 1 вела передачу, они вновь становятся синхронны. Кроме того, коллизия невозможна в течение ЗИ. Поэтому вероятность первой коллизии для станции 3 будет следующей: 
\begin{equation}
\pc{0}{3} = \tst{1}{3}\frac{1-p_d}{W_0-d-1} \sum\limits_{i = 1}^{W_0-d-1} (1- (1-\tst{1}{3})^i) = (1-p_d)\left[\tst{1}{3} - \frac{(1-\tst{1}{3})(1-(1-\tst{1}{3})^{W_0-d-1})}{W_0-d-1}\right]
\end{equation}

Рассмотрим это выражение более подробно. Если станция 3 выбрала слот для передачи под номером $d+i$, то станция 1 может начать передавать раньше, если она выберет у себя слот под номером $i-1$ или менее. Т.~е.

\begin{tabular}{|>{$}l<{$}| >{$}l<{$}|}
\hline
\text{Слот станции 3} &\text{Вероятность передачи станции 1} \\
\hline
d+1 & \tst{1}{3} \\
d+2 & 1-(1-\tst{1}{3})^2 \\
d+3 & 1-(1-\tst{1}{3})^3 \\
\dots &\dots \\
W_0-1 &1-(1-\tst{1}{3})^{W_0-d-1} \\
\hline
\end{tabular}

Таким образом, вероятность
\begin{equation}
\notag
\frac{1-p_d}{W_0-d-1} \sum\limits_{i = 1}^{W_0-d-1} (1- (1-\tst{1}{3})^i)
\end{equation}
есть вероятность того, что станция 1 начнет передачу раньше станции 1, в результате чего станции снова станут синхронизированы. Остается домножить это выражение на $\tst{1}{3}$, чтобы получить вероятность первой коллизии станции 3.

Заметим, что величина $\pc{0}{3}$ есть вероятность следующего события: во время отсчета слотов отсрочки станция 3 выбрала слот больший чем $d$, причем станция 1 успела передать до начала слота $d$ станции 3, после чего станция 3, отсчитав до слота под номером $d$ попала в коллизию со станцией 1. Заметим, что коллизии может не быть по следующим причинам: либо станция 3 выбрала один из слотов $0, \dots, d$, либо станция 1 не успела влезть, когда станция 3 выбрала слот больший, чем $d$. Поэтому вероятность того, что нет коллизии при первой передаче, при условии, что станция 3 выбрала слот больший, чем $d$, равна $1 -\pc{0}{3} - p_d$, т.е.

\begin{tabular}{l l}
$p_d$ 		&станция 1 не вышла за пределы слота $d$ \\
$\pc{0}{3}$	&станция 1 вышла за пределы слота $d$, причем станция 3 успела передать \\
$1-\pc{0}{3}-p_d$ &станция 1 вышла за пределы слота $d$, но станция 3 не успела передать \\
\end{tabular}

После первой же коллизии станции, очевидно, будут синхронны, поэтому
\begin{equation}
\notag
\pc{i}{3} = \tst{1}{3}, i = \overline{1,N_r-1}.
\end{equation}
$\tst{1}{3}$ и $\tst{3}{1}$ находятся из (4) и (5), причем компоненты (4) при нахождении  несколько меняются из-за того, что станция 1 не отсчитывает интервал отсрочки внутри ЗИ:
\begin{gather}
\notag
f_3 = 1 - p_d + \sum\limits_{i=1}^{N_r-1}\prod\limits_{j=0}^{i-1}\pc{j}{3}, \\
\notag
w_3 = (1-p_d)\frac{W_0-d-1}{2} + \sum\limits_{i=1}^{N_r-1}\frac{W_i-1}{2}\prod\limits_{j=0}^{i-1}\pc{j}{3}.
\end{gather}
Здесь нас интересуют только те попытки передачи и те  отсчеты слотов, которые происходили вне ЗИ, и поэтому могли прервать отсчет слотов станции 1.

\subsection{Случай 3}
Рассмотрим следующий случай (рис. 6), при котором станции 1 и 4 вовсе не «слышат» друг друга (т.е. находятся вне CS-области).

Как показано на рис. 7, в случае одновременной передачи станций 1 и 3 коллизия произойдет на станции 2, но не на станции 4, поскольку она скрыта от станции 1, т.е. станция 3 вообще не испытывает коллизий. Поэтому станции 3 и 4 успешно продолжат обмен кадрами. Это значит, что с точки зрения станции 1 длительность коллизии равна времени  успешной передачи.  Далее используем общий метод с $\pc{i}{3} = 0$ для всех $i$ и $\tst{3}{1}=\frac{2}{W_0+1}$, поскольку конкурентное окно станции 3 всегда минимально.


$\lc{1} = l_s$, так как длительность коллизии равна длительности передачи станции 3.

$\lc{3} = 0$, так как нет коллизий.

$\la{1} = l_s$

$\la{3} = l_s$

$\pc{i}{3} = 0$, $i = \overline{0,N_r-1}$.

$\pc{i}{1} = \tst{3}{1}$, $i = \overline{0,N_r-1}$.

$\tst{1}{3} = \frac{f_1(\tst{3}{1})}{w_1(\tst{3}{1})+f_1(\tst{3}{1})}$

\subsection{Случай 4}
В этом случае станции 1 и 4 не слышат друг друга вовсе, а станции 1 и 3 так же, как и станции 2 и 4 находятся в пределах CS-области друг друга (см. рис 8).

Данный случай схож со случаем 2. После передачи кадра данных станцией 3, станция 1 не сможет декодировать кадр данных и будет вынуждена ждать время EIFS до возобновления отсчета слотов отсрочки, в то время как станции 3 необходимо ждать меньший интервал SIFS+ACK+DIFS. Поэтому аналогично случаю 2 появляется ЗИ, во время которого станция 3 может начать свою передачу, не испытывая конкуренции со стороны станции 1 (см. рис. 9).

Длина ЗИ в этом случае равна $d = \left\lfloor \hat{d} = \frac{EIFS - (SIFS + ACK + DIFS)}{\sigma}\right\rfloor$ слотов, и формула 
\begin{equation}
\la{1} = l_s + \hat{d}\sigma + \frac{p_d}{1-p_d}(l_s+\frac{d}{2}\sigma)
\end{equation}
верна с этим новым значением $d$. Однако в отличие от случая 2 станции 1 и 3 становятся синхронны только после успешной  передачи станции 1, но не после коллизии. Более того, аналогично случаю 3, если станции начнут передавать одновременно, коллизия произойдет только на станции 2, но не на станции 4. Поэтому показатели производительности для соединения $3\rightarrow 4$ находятся с помощью формул (1), (2), (3), (9) и  для всех i . Как и в случае 3, конкурентное окно станции 3 всегда минимально, поэтому ее вероятность передачи равна $\tst{3}{1} = \frac{2}{W_0-d+1}$.

Заметим, что в отличие от случая 2 лишь одна коллизия возможна для соединения $1\rightarrow 2$. В самом деле, пусть на станции 2 произошла коллизия, тогда станция 3 в любом случае завершит свою передачу успешно и станции 1 и 3 станут асинхронны, поэтому вероятность последующих коллизий равна нулю. Таким образом, вероятность  первой коллизии для станции 1 вычисляется согласно (8), в то время как  для , а также $\prej{1} = 0$. Для соединения $1 \rightarrow 2$ длительность коллизионного слота равна времени успешной передачи станции 3, включая возможные дополнительные передачи из-за наличия ЗИ, т.е. $\lc{1} = \la{1}$. Вероятность передачи $\tst{1}{3}$ находится с помощью общего метода.

\begin{table}[h]
\caption{Сводка формул}
\begin{center}
\begin{tabular}{>{$}l<{$} >{$}l<{$} >{$}l<{$}}
d 				&= &\left\lfloor \hat{d} = \frac{EIFS - (SIFS + ACK + DIFS)}{\sigma}\right\rfloor \\
\la{1}			&= &l_s + \hat{d}\sigma + \frac{p_d}{1-p_d}(l_s+\frac{d}{2}\sigma) \\
\lc{1} 	&= &\la{1} \\
\tslot{1} &= &(1-\tst{3}{1})\sigma + \tst{3}{1}\la{1} \\
\at{i}{1} 	&= &\frac{W_i-1}{2}\tslot{1} + (1-\pc{i}{1})l_s + \pc{i}{1}\lc{1} \\
\pc{0}{1}		&= &\tst{3}{1} \frac{1}{W_0}\sum\limits_{b = 0}^{W_0-1}(1-\tst{3}{1})^b \\
\pc{i}{1} 	&= &0\\
\tst{1}{3}	&= &\frac{f_1(\tst{3}{1})}{w_1(\tst{3}{1})+f_1(\tst{3}{1})} \\
\prej{1} 	&= &0\\
\la{3} 		&= &l_s \\
\lc{3} 		&= &0 \\
\tslot{3}	&= &(1-\tst{1}{3})\sigma + \tst{1}{3}\la{3} \\
\at{0}{3} 	&= &l_s \\
\pc{i}{3} 	&= &0\\
\tst{3}{1}	&= &\frac{2}{W_0-d+1}\\
\prej{3} 	&= &0\\
\end{tabular}
\end{center}
\end{table}

\textbf{Вопрос} Здесь также как и в случае 2 вероятность коллизии $\pc{0}{1}$ должна зависеть от того, синхронны ли станции или нет.

\subsection{Случай 5}
В  случае 5 лишь станции 2 и 3 слышат друг друга (см. рис 10). Поскольку станция 1 вовсе не слышит станцию 3, то она может начать свою передачу в любое время, вне зависимости от поведения станций 3 и 4.  В частности, отсчет DIFS станцией 3 может быть прерван посылкой ответного кадра CTS станцией 2.

В данном случае удобно интерпретировать $\tst{1}{3}$ как вероятность того, что станция 2 начнет отвечать кадром CTS во время отсчета станцией 3 интервала отсрочки или DIFS. Длительность виртуального слота станции 3, заполненного успешной  передачей станции 1, равна $\la{3} = l_s - RTS - SIFS$, поскольку станция 3 не слышит кадр RTS, посланный станцией 1. 
Так как станция 2, использующая механизм виртуального прослушивания канала, не реагирует на приход кадра RTS от станции 1 в интервале от начала передачи RTS станцией 3 до окончания передачи ACK станцией 4, прерывание DIFS возможно только спустя SIFS после его начала. Длительность интервала, в течение которого станция 1 может прервать отсчет DIFS станции 3, равна $d = \left\lfloor \hat{d}= \frac{DIFS - SIFS}{\sigma}\right\rfloor$ слотам.

	Аналогично случаю 3, передача станции 3 всегда успешна и ее конкурентное окно всегда минимально. Поэтому 
\begin{equation}
\Est{3} = l_s - DIFS + SIFS + \left[ d + \frac{W_0-1}{2} \right]
\tslot{3}
\end{equation}

Теперь  рассмотрим соединение $1\rightarrow 2$. Передача станции 1 будет неудачной в случае, если она началась либо за время (RTS+SIFS) до передачи станции 3, либо внутри интервала  после начала передачи станции 3 (см. рис 11). Цикл передачи станции 3 состоит из самой передачи и интервала отсрочки. Станция 1 может равновероятно начать передачу кадра RTS в любой момент внутри цикла передачи станции 3 за исключением времени, когда станция 3 замораживает счетчик отсрочки из-за передачи станции 1. С учетом этого исключения длина одного цикла равна . Тогда находим вероятность коллизии как следующее отношение:
		(11)

и оцениваем показатели производительности соединения 1->2, используя общий метод и учитывая, что в этом случае  и . Для нахождения  воспользуемся (4).

\begin{table}[h]
\caption{Сводка формул}
\begin{center}
\begin{tabular}{>{$}l<{$} >{$}l<{$} >{$}l<{$}}
d 			&= &\left\lfloor \hat{d} = \frac{DIFS - SIFS}{\sigma}\right\rfloor +\\
\la{1}		&= &\sigma + \\
\lc{1} 		&= &RTS+T_c+ \\
\tslot{1} 	&= &(1-\tst{3}{1})\sigma + \tst{3}{1}\la{1} = \sigma ? \\
\at{i}{1} 	&= &\frac{W_i-1}{2}\tslot{1} + (1-\pc{i}{1})l_s + \pc{i}{1}\lc{1} \\
\pc{i}{1}	&= &p_c = \frac{SIFS + l_s - DIFS}{l_{cycle}^{(3)}}+ \\
l_{cycle}^{3} &= &l_s + \frac{W_0-1}{2}\sigma +\\ 
\tst{1}{3}	&= &\\
\prej{1} 	&= &p_c^{N_r-1} ?\\
\la{3} 		&= &l_s - RTS - CTS+ \\
\lc{3} 		&= &0 ?\\
\tslot{3}	&= &(1-\tst{1}{3})\sigma + \tst{1}{3}\la{3} ? \\
\at{0}{3} 	&= &\frac{W_0-1}{2}\tslot{3} + l_s ? \\
\pc{i}{3} 	&= &0  ?\\
\tst{3}{1}	&= &0 ?\\
\prej{3} 	&= &0 ?\\
\end{tabular}
\end{center}
\end{table}

\subsection{Случай 6}
В случае 6 станции 2 и 3 находятся в CS-области друг друга. Остальные станции полностью скрыты друг от друга (см. рис. 12). Аналогично случаю 5 станция 1 может начать свою передачу независимо от поведения станций 3 и 4, поэтому отсчет DIFS станции 3 а также ее интервала отсрочки может быть прерван ответом станции 2.

Обмен кадрами RTS/CTS не защищает передачу данных станции 1, поскольку  механизм виртуального прослушивания в этом случае не работает. Таким образом, вероятность коллизии $p_c$ нужно разделить на два компонента: вероятность $p_{cR}$ коллизии кадра RTS и вероятность $p_{cD}$ коллизии кадра данных при условии, что передача RTS была успешна. Ввиду неработоспособности механизма виртуального прослушивания, станция 2 отвечает кадром CTS, даже если станция 1 послала свой RTS сразу после окончания передачи кадра данных станцией 3. Поэтому в (10) исключим SIFS из суммы и переопределим $d$: $d = \left\lfloor \frac{DIFS}{\sigma} \right\rfloor$ (Внимание, читерство: здесь полагается, что $RTS = ACK$).  Как будет показано далее, если DIFS или интервал отсрочки станции 3 будет прерван кадром CTS станции 2, вслед за которым последует интервал EIFS (поскольку станция 3 не может декодировать кадр), затем интервал отсрочки станции 3 также будет прерван кадром ACK станции 2 с вероятностью  $1-p_{cD}$ (см. рис. 13). Следовательно, $\la{3} = CTS + EIFS + (1-p_{cD})(ACK + EIFS)$. Далее используем (10) для нахождения $\Est{3}$.

Теперь найдем вероятности коллизии для станции 1. Как и в случае 5, введем понятие цикла передачи как интервала времени между окончаниями двух последовательных передач кадров данных станцией 3. Пусть $t = 0$  соответствует моменту начала цикла передачи.  Передача кадра RTS станции 1 успешна, если она начинается в момент , где  и  ­­­­ – ­­время отсрочки станции 3. Очевидно, средняя длина интервала  успеха RTS равна . Тогда~\eqref{case5:pc1} изменяется следующим образом:
\begin{equation}
p_{cD} = \frac{l_{cycle}^{(3)} - l_s^{RTS}}{l_{cycle}^{(3)}} = \frac{RTS + l_s - SIFS- ACK - DIFS}{l_{cycle}^{(3)}},
\end{equation}
где $l_{cycle}^{(3)} = l_s + \frac{W_0-1}{2}\sigma$.

Даже если станция 1 успешно передает свой кадр RTS, следующий за ним кадр данных может быть искажен последующей передачей кадра RTS станцией 3, и вероятность коллизии данных зависит от интервала отсрочки станции 3. 

Если станция 1 начала свою передачу в момент времени , она завершит передачу кадра данных в момент времени
\begin{equation}
\notag
t_1^c = t_1 + RTS + 2 \times SIFS + CTS + DATA.
\end{equation}
Кадр CTS станции 2 прерывает или DIFS станции 3, если $t_1 < \Delta - SIFS$, или ее интервал отсрочки. Следовательно, станция 3 закончит отсчитывать интервал отсрочки в момент времени 
\begin{equation}
\notag
t_3^c = t_1 + RTS + SFIS +CTS + EIFS + b\sigma - \max\{0, t_1 - \Delta + SIFS\}.
\end{equation}
Очевидно, что кадр данных станции 1 будет передан успешно, если $t_3^c \geqslant t_1^c$. Возможны два случая:
\begin{enumerate}
\item Если $t_1 < \Delta - SIFS$, то получим, что
\begin{equation}
\notag
t_1 < \Delta - SIFS \text{ и } b > b_0 = \left\lfloor\frac{DATA + SIFS - EIFS}{\sigma}\right\rfloor.
\end{equation}
\item Если $t_1 \geqslant \Delta - SIFS$, то получим, что
\begin{equation}
\notag
t_1 \leqslant \Delta_1 + b\sigma.
\end{equation}
Должно быть, что $\Delta_1 + b\sigma \geqslant \Delta - SIFS$, откуда получаем ограничение на $b$:
\begin{equation}
\notag
b > b_0 = \left\lfloor\frac{DATA + SIFS - EIFS}{\sigma}\right\rfloor.
\end{equation}
\end{enumerate}
Объединяя полученные данные, получаем, что кадр данных станции будет успешно передан, если
\begin{equation}
\notag
b > b_0 = \left\lfloor \frac{DATA+SIFS-EIFS}{\sigma}\right\rfloor \text{ и } t_1 \leqslant \Delta_1 + b\sigma,
\end{equation}  
где
\begin{equation}
\notag
\Delta_1 = EIFS + ACK + DIFS - SIFS -RTS - DATA.
\end{equation}
Таким образом,
\begin{equation}
\notag
p_{cD} = 1 - \frac{1}{W_0 l_s^{RTS}}\sum\limits_{b = b_0+1}^{W_0-1} (\Delta_1+b\sigma) = 1 - \frac{W_0-1-b}{W_0 l_s^{RTS}}\left(\Delta_1 + \sigma\frac{W_0+b_0}{2}\right),
\end{equation}
где 
\begin{equation}
\notag
\frac{1}{W_0 l_s^{RTS}}\sum\limits_{b = b_0+1}^{W_0-1} (\Delta_1+b\sigma)
\end{equation}
есть вероятность успешной передачи кадра данных. Действительно, при условии, что станция 1 может начать передачу в любой момент времени, величина $\frac{\Delta_1+b\sigma}{l_s^{RTS}}$ при фиксированном $b > b_0$, есть вероятность того, что будет успешно передан кадр данных (при этом успешно передан кадр RTS). Проведем суммирование по множеству значений $b$, при которых успешно доставляется кадр данных. Тогда и получим вероятность успешной передачи кадра данных.

Поскольку число попыток передачи кадра RTS перед каждой передачей кадра данных и число попыток передачи кадра данных ограничены соответственно пределами $N_r$ и  $N_d$, вероятность отказа вычисляется по следующей формуле:
\begin{equation}
\notag
\prej{1} = p_{cD}^{N_d}[1-p_{cR}^{N_r}]^{N_d} + \sum\limits_{j=1}^{N_d}[1-p_{cR}^{N_r}]^{j-1}p_{cR}^{N_r}p_{cD}^{j-1}
\end{equation}

\begin{table}[h]
\caption{Сводка формул}
\begin{center}
\begin{tabular}{>{$}l<{$} >{$}l<{$} >{$}l<{$}}
d 			&= &\left\lfloor \frac{DIFS}{\sigma} \right\rfloor\\
\la{1}		&= &\\
\lc{1} 		&= &\\
\tslot{1} 	&= &\\
\at{i}{1} 	&= &\\
\pc{i}{1}	&= &\\
l_{cycle}^{3} &= &\\ 
\tst{1}{3}	&= &\\
\prej{1} 	&= &\\
\la{3} 		&= &CTS + EIFS + (1-p_{cD})(ACK+EIFS)\\
\lc{3} 		&= &\\
\tslot{3}	&= &\\
\at{0}{3} 	&= &\\
\Est{3}		&= &l_s + SIFS + \left[d + \frac{W_0-1}{2}\right] \tslot{3} \\
\pc{i}{3} 	&= &\\
\tst{3}{1}	&= &\\
\prej{3} 	&= &\\
\end{tabular}
\end{center}
\end{table}
}