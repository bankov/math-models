\newcommand{\Tob}{T_\text{об}}
\newcommand{\prej}{p_{rej}}
\newcommand{\SIFS}{\text{SIFS}}
\newcommand{\EIFS}{\text{EIFS}}
\newcommand{\ACK}{\text{ACK}}
\newcommand{\DIFS}{\text{DIFS}}
\newcommand{\CTS}{\text{CTS}}
\newcommand{\RTS}{\text{RTS}}

\section{DCF}
Модель Ad-Hoc сети.  2 метода: простой и RTS/CTS.

--- TODO: тут рисунок ---
\[ b \in (0, CW_{min}) \]

Предположения:
\begin{itemize}
\item станции работают в насыщении
\item все станции слышат друг-друга
\item потери пакетов не связаны со случайным шумом (идеальный канал)
\end{itemize}

Обозначения:
\begin{itemize}
\item $N$ --- количество станций
\item $S$ --- пропускная способность (доля времени, занимаемая успешной передачей данных)
\item $\Tob$ --- среднее время обслуживания
\item $R$ --- количество повторных попыток передач
\item $\prej$ --- вероятность потери пакета
\item $\bar{d}$ --- средняя длина пакета
\end{itemize}

Необходимо найти $S$, $\Tob$, $\prej$.

С точки зрения одной станции
\[
S = \frac{(1 - \prej) \bar{d}}\Tob
\]

Виртуальный слот --- это интервал времени между последовательными изменениями счетчика отсрочки.  Виртуальный слот может быть пустым, успешным и коллизионным.

\begin{itemize}
\item $p_e$ --- вероятность пустого слота
\item $p_s$ --- вероятность успешного слота
\item $p_c$ --- вероятность коллизионного слота
\end{itemize}

Длительность пустого, успешного и коллизионного слота $\sigma$, $T_s$ и $T_c$ соответственно.

С точки зрения всей системы в целом
\[ S = \frac{p_s \bar{d}}{T_{slot}}, \]
где $T_slot = p_e \sigma + p_s T_s + p_c T_c$ --- средняя длительность виртуального слота.

Пусть $\tau$ --- вероятность того, что станция выбрала текущий слот для передачи.

Вероятность того, что пакет имеет длину $l$ равна $d_l$, $\bar{d} = \sum d_l \cdot l$.

$D_l = \sum\limits_{i = 1}^{l}$ --- функция распределения


\begin{align*}
p_e &= (1 - \tau)^N \\
p_s &= N \tau (1 - \tau)^{N - 1} \\
p_c &= 1 - p_e - p_s
\end{align*}

Порог RTS/CTS $\bar{p}$ --- если длина пакета превышает его, то используется RTS/CTS, иначе базовый метод.

\[
T_s = (1 - D_{\bar{p}}) (\RTS + \CTS + 2 \cdot \SIFS) +
\ACK + \SIFS + \DIFS + \sum_l d_l t_d(l),
\]
где $t_d(l) = \frac{H + 8 \cdot l}{V}$ --- время, $H$ --- header, $l$ --- число байт, $V$ --- битовая скорость канала

--- TODO: рисунки ---

\[
T_c = \sum{n=2}^N \pi_n \cdot t_c(n),
\]
где
\begin{itemize}
\item $\pi_n$ --- вероятность того, что $n$ станций выбрали данный слот для передачи, если известно, что произошла коллизия,
\item $t_c(n)$ --- средняя длительность коллизии
\end{itemize}

\[
\pi_n = {N \choose n} \frac{\tau^n (1 - \tau)^{N - n}}{p_c}, n \geq 2
\]

Далее рассматриваем три типа коллизий: RTS-RTS, RTS-DATA, DATA-DATA

\begin{enumerate}
\item RTS-RTS

\begin{itemize}
\item вероятность такой коллизии --- $(1 - D_{\bar{p}})^n$
\item длительность коллизии --- $\RTS + \EIFS = t_c(n)$
\end{itemize}

\item DATA-DATA
\[ F(\max(\xi_1, ..., \xi_N) < x) = (F_\xi (x))^N \]

Вероятность того, что максимум равен $l$: $(D_l)^n - (D_{l-1})^n$.

\[
t_c(n) = \sum_{l < \bar{p}} ((D_l)^n - (D_{l-1}^n) t_d(l) + \EIFS
\]

\item RTS-DATA

$(1 - D_{\bar{p}} + D_l)^n - (1 - D_{\bar{p}})^n$ --- какой-то из пакетов по длине меньше $l$.
$(1 - D_{\bar{p}} + D_{l-1})^n - (1 - D_{\bar{p}})^n$ --- какой-то из пакетов по длине меньше $l - 1$.

Какой-то из пакетов по длине равен $l$:
\begin{align*}
(1 - D_{\bar{p}} + D_l)^n  - (1 - D_{\bar{p}} + D_{l-1})^n &= \\
(1 - D_{\bar{p}} + D_{l - 1} + d_l)^n - (1 - D_{\bar{p}} + D_{l - 1})^n &= \\
\sum_{i = 1}^n {{n \choose i} (d_l)^i (1 - D_{\bar{p}} + D_{l - 1})^{n - i}} &
\end{align*}

${n \choose i} (d_l)^i (1 - D_{\bar{p}} + D_{l - 1})^{n - i}$ --- $i$ пакетов длины $l$, остальные меньше $l$ или больше $\bar{p}$.

В результате:
\[
t_c(n) = (1 - D_{\bar{p}})^n \RTS + \sum_{l = 1}^{\bar{p}} ((1 - D_{\bar{p}} + D_l)^n - (1 - D_{\bar{p}} + D_{l - 1})^n) \cdot t_d(l) + \EIFS
\]
\end{enumerate}

$\prej = (1 - (1 - \tau)^{N - 1})^{R + 1}$ --- Вероятность $R + 1$ раз выбрать коллизионный слот.

Ищем $\tau$.  Состояние станции описывается парой чисел $(s, b)$, где $s \in 0, ..., R$ --- стадия отсрочки, $b \in 0, ..., CW - 1$ --- значение счетчика отсрочки, $CW_{min} = W_0 - 1$.

--- TODO: рисунок марковской цепи ---

\begin{align*}
\pi_{1,W_1-1}   &= \frac{p}{W_1} \pi_{0,0} \\
\pi_{1,W_1-2}   &= \frac{p}{W_1} \pi_{0,0} + \frac{p}{W_1} \pi{00} = \frac{2p}{W_1}\pi_{0,0} \\
... \\
\pi_{1,W_1-k}   &= \frac{pk}{W_1} \pi_{0,0} \\
\pi_{1,0}       &= p \pi_{0,0} \\
\pi_{2,0}       &= p^2 \pi_{0,0} \\
... \\
\pi_{R,0}       &= p^R \pi_{0,0} \\
\pi_{0,W_0 - 1} &=     \frac{\pi_{R,0}}{W_0} + \frac{1 - p}{W_0} \sum_{i=1}^R \pi_{i,0} = \\
                &=     \frac{\pi_{R,0}}{W_0} + \frac{1 - p}{W_0} \sum_{i=1}^R \pi^i \pi_{0,0} = \\
                &= p^R \frac{\pi_{0,0}}{W_0} + \frac{1 - p}{W_0} \pi_{0,0} \frac{1 - p^R}{1 - p} = \\
                &= \frac{\pi_{0,0}}{W_0} \\
\pi_{0,W_0 - 2} &= \frac{2 \pi_{0,0}}{W_0}, ..., \pi_{0,W_0 - k} = \frac{k \pi_{0,0}}{W_0}
\end{align*}

Условие норимировки:
\[
\sum_{i=0}^R (\pi_{i,0} + ... + \pi_{i,W_i-1}) =
\sum_{i=0}^R p^i \frac{1 + W_i}{2} \pi_{0,0} =
(\sum_{i=0}^R p^i \frac{1 + W_i}{2}) \pi_{0,0} = 1
\]

Отсюда
\[
\pi_{0,0} = \frac{1}{\sum_{i=0}^R \frac{1 + W_i}{2} p^i} = (\sum_{i=0}^R \frac{1 + W_i}{2} p^i)^{-1}
\]

\[
\prej = p \pi_{R,0} = p^{R+1} \pi_{0,0} = p^{R+1} (\sum_{i=0}^R \frac{1 + W_i}{2} p^i)^{-1}
\]

\[
\tau = \sum_{i = 0}^R \pi_{i,0} = \pi_{0,0} \sum_{i=0}^R p^i = (\sum_{i = 0}^R \frac{1 + W_i}{2} p^i)^{-1} \frac{1 - p^{R+1}_{1-p}} (\sum_{i=0}^R \frac{1 - p^{R+1}}{1-p}) = 1 - (1 - \tau)^{N - 1}
\]

Из полученного равенства находим $\tau$.

$\tau = \frac{F}{\bar{W}}$, где $F$ --- среднее число попыток передач, $\bar{W}$ --- среднее число виртуальных слотов.

\[
F = \sum_{i=1}^{R+1} p^{i-1} = \frac{1-p^{R+1}}{1 - p}
\]
\[
\bar{W} = \sum_{i=0}^R \frac{W_i+1}{2} p^i
\]
